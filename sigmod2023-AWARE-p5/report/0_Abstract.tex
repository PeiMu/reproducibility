

% State the problem
Compression is an effective technique for fitting data in available memory, reducing I/O, and increasing instruction parallelism. While data systems primarily rely on lossless compression, modern machine learning (ML) systems exploit the approximate nature of ML and mostly use lossy compression via low-precision floating- or fixed-point representations. The resulting unknown impact on learning progress, and model accuracy, however, create trust concerns, that require trial and error, and are problematic for declarative ML pipelines.
% Say why it's an interesting problem
Given the trend towards increasingly complex, composite ML pipelines---with outer loops for hyper-parameter tuning, feature selection, and data cleaning/augmentation---it is hard for a user to infer the impact of lossy compression. Sparsity exploitation is a common lossless scheme used to improve performance without this uncertainty. Evolving this concept to general redundancy-exploiting compression is a natural next step. Existing work on lossless compression and compressed linear algebra (CLA) enable such exploitation to a degree, but face challenges for general applicability.
% Say what your solution achieves
In this paper, we address these limitations with a workload-aware compression framework, comprising a broad spectrum of new compression schemes and kernels. Instead of a data-centric approach that optimizes compression ratios, our workload-aware compression summarizes the workload of an ML pipeline, and optimizes the compression and execution plan to minimize execution time.
% Say what follows from your solution
On various micro benchmarks and end-to-end ML pipelines, we observe improvements for individual operations up to \numprint{10000}x and ML algorithms up to \numprint{6.6}x compared to uncompressed operations.
